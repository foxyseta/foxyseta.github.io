\documentclass{letter}
\signature{Stefano}

\usepackage[italian]{babel}
\usepackage[utf8]{inputenc}

\begin{document}
\begin{letter}{}
\opening{A G.R.,}
non ero "interessato" a te: mi sentivo minacciato da te. Ho contato le
somiglianze fra noi prima di confrontarci in una infantile rivalità a senso
unico, riscoprendo ogni mia peculiarità (positiva e non) come pallida imitazione
di un tuo pregio.

Ti vedevo solo e ti ho presentato a degli amici, ma fra loro mi hai battuto in
popolarità. Non sei legato al bisogno di ostentazione delle proprie capacità che
mi caratterizza, eppure in aula ottieni risultati buoni quanto i miei. Le
persone apprezzano anche il tuo fisico, però non ti senti costretto al ruolo di
troia del gruppo. Non punti come me alla discoteca dedicatata al puro
divertimento, bensì a quella inserita in un circolo attivista e dedito al
volontariato. Hai scelto consciamente di abbandonare la tua fede, e resti
comunque una persona più integra di me.

È pena quella che provi per me? Silenziosamente, ti congratuli mai fra te e te
per la tua manifesta superiorità? Non avrei comunque modo di saperlo: ogni cosa
che fai uscire dalla tua bocca è gentile e rispettosa e condivisa e apprezzata.

Vorrei avvicinarti per raccontarti che fra noi ci sono anche somiglianze, e
tentare forse così di appiattire, anche se solo in parte, il dislivello che
sento. So già, però, come andrebbe a finire: mi smentiresti, rivelandomi un
qualche grande fattore discriminante che ci differenzia. È un \textit{topos}
narrativo che, per quanto trito e ritrito, porta sempre il protagonista ad
affermare la propria superiorità morale rispetto all'antagonista. Se in esso
dovessi interpretare la parte di quest'ultimo, verrei bollato anche io come
individuo inferiore.
\closing{Definitivamente.}
\end{letter}
\end{document}
