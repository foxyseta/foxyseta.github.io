\documentclass{article}

\usepackage[american, italian]{babel}

\title{Realtà estese e noi, loro abitanti}
\author{Gruppo FUCI di Bologna}
\date{\today}

\begin{document}
\maketitle
La futurologia, e cioè la previsione dei prossimi sviluppi della tecnologia e
del suo impatto nel quotidiano, è nota per fare cilecca. Alcune invenzioni oggi
essenziali furono, ai tempi della loro invenzione, estremamente sottovalutate:
``L'automobile è una moda passeggera, una novità del momento, mentre i cavalli
continueranno a essere usati da tutti'' diceva il Presidente della
\emph{Michigan Savings Bank} nel 1903. Gli ambiti di applicazione di altre
tecnologie nella vita pratica, al contrario, sono stati enormemente sovrastimati prima del tempo:
``Senza dubbio avremo aspirapolveri a energia nucleare entro circa 10 anni''
garantiva un \emph{CEO} del settore nel 1955. Una breve ricerca in rete vi
convincerà che questi non sono casi isolati, né i più comici dell'ambito. Non
fatevi ingannare: queste cantonate non sono certo esclusive dell'età
contemporanea. Se vi è capitato di studiare un po' di storia della filosofia
alle superiori, di certo conoscerete l'avversione di Socrate per la
comunicazione scritta: guai a chi perde le giornate con il naso immerso nella
lettura! Quella nefasta attività atrofizza la mente, causa confusione fra realtà
e finzione, e scoraggia le persone dal parlarsi faccia a faccia! I giovinastri e
le loro diavolerie moderne...

Insomma, parlare delle tecnologie del futuro è spesso un fiasco. Sia chiaro,
quindi, che quando ci interroghiamo sulla realtà estesa (o XR, da
\begin{otherlanguage}{american}\emph{extended reality}\end{otherlanguage})
stiamo, volenti o nolenti, parlando di presente. Essa include strumenti che
hanno già dato prova del proprio impatto sulla vita delle persone: nella
fattispecie, la realtà aumentata (AR,
\begin{otherlanguage}{american}\emph{augmented reality}\end{otherlanguage}),
responsabile del fenomeno di massa di \emph{Pokémon GO} del 2016, la realtà
virtuale (VR,
\begin{otherlanguage}{american}\emph{virtual reality})\end{otherlanguage}, uno
dei tanti mezzi già aperti al publico (stra)pagante per fruire del ``metaverso''
di Zuckerberg, e la realtà mista (MR, \begin{otherlanguage}{american}\emph{mixed
reality}\end{otherlanguage}), che racchiude tutti gli ibridi posti in posizione
intermedia fra le precedenti.

Gli esempi di cui sopra sono stati ripresi da una mozione di indirizzo
presentata quest'anno dal nostro gruppo. Essa, trattando un argomento in buona
parte tecnologico, è stata definita una ``mozione di opposizione'', con chiaro
rimando al lessico della politica, molto familiare a diversi fra noi fucini.
Anche nel nostro caso il compito dell'opposizione è quello di presentare
un'alternativa, e in particolare una strada tematica diversa e meno battuta, che
non vuole sostituirsi ai discorsi della maggioranza, ma fornirvi una seconda
voce, strumento necessario (anche se non sufficiente) per aprire un dialogo. I
mondi della scuola, dell'università, della ricerca e perfino della politica
italiane sono spesso quantitativamente e qualitativamente carenti di
multidisciplinarietà e interdisciplinarietà. Come studentesse e studenti
dell'università, siamo forse fra le più indicate e i più indicati per questo
scopo.

Non c'è infatti motivo di lasciare il monopolio di questi argomenti alle
ingegnere e agli ingegneri o alle informatiche e agli informatici che
attualmente se ne occupano. Le giuriste e i giuristi potranno discutere su come
regolamentare la protezione dei dati personali, nonché su cosa dovrebbe o non
dovrebbe essere concesso a persone e aziende in questo nuovo contesto. Le
filosofe e i filosofi dovranno fornire ai giuristi le basi di etica necessarie,
per poi sbizzarrirsi con nuove varianti dell'argomento del cervello nella vasca
di Putnam. Si ritroveranno costrette e costretti a rivedere molte teorie di
filosofia della conoscenza. Le psicologhe e gli psicologi, quindi, si sentiranno
in dovere di portare sul tavolo i propri risultati sulla psicologia cognitiva,
ma avranno anche cura di verificare i potenziali effetti della realtà estesa
sulla salute emotiva. Le studiose e gli studiosi di statistica, infine, si
assicureranno che gli studi citati nelle discussioni siano rilevanti, e che le
loro interpetazioni non siano fallaci. È un gioco in cui tutte e tutti portano
qualcosa di proprio in campo. Come mai, però, riversarvi tante energie?

Interrogarci sugli strumenti che usiamo dice qualcosa su noi stessi. Il modo in
cui li usiamo racconta i nostri scopi. L'entusiasmo che ci travolge per essi
parla delle nostre aspirazioni. I timori che proviamo allo scontrarsi con questi
mezzi sono memento delle minacce da noi più sentite. Cosa ci affascina della
realtà estesa rispetto a quella ``analogica''? Quella che inseguiamo è una fuga
dal mondo attuale o solo una sua integrazione? Quali aspetti siamo restii ad
abbandonare? Perché potremmo o dovremmo essere esistanti nell'abbracciare questo
cambiamento? Porsi questi quesiti non significa avvicinarsi all'ennesima moda
fantascientifica, ma rispondere a qualche domanda su noi che la abitiamo, sia
come individui che come società.

Chi invece alle domande profonde preferisce un gesto umano immediato e concreto,
sarà felice di scoprire che la dott.ssa Melissa Wong, ostetrica e ginecologa di
professione, è riuscita già nel 2020 a concludere con successo uno studio dai
risultati più che incoraggianti sull'uso della realtà virtuale per distrarre le
sue pazienti dai dolori causati dalle contrazioni del parto. L'effettiva qualità
umana delle relazioni instaurate durante una conversazione nella realtà estesa
è invece al momento ancora difficile da quantificare: pur esistendo studi
precedenti che ne elogiano i benefici specialmente per le persone introverse,
essi sono spesso e volentieri sovvenzionati dai grandi marchi interessati a
mostrare le proprie tecnologie come tutto fuorché alienanti e distopiche.

La questione rimane quindi aperta, e ridurne la soluzione a una monolitica e
monosillabica risposta che marchi la realtà estesa come ``buona'' o ``cattiva''
sarebbe a dir poco semplicistico. La giusta strategia sarebbe quindi un
approccio destrutturante, che scomponga questa tecnologia nelle sue varie
applicazioni per poi esprimere pareri indipendenti su ciascuna di esse. Un
lavoro lungo e non banale, sì, ma che è bene sia effettuato prima di ritrovarci
polarizzati in una fazione ciecamente tradizionalista e in una ottusamente
tecnomane.

\end{document}
