%%%%%%%%%%%%%%%%%%%%%%%%%%%%%%%%%%%%%%%%%%%%%%%%%%%%%%%%%%%%%%%%%%%%%%%%%%%%%%%%
% Medium Length Professional CV
% LaTeX Template
% Version 2.0 (8/5/13)
%
% This template has been downloaded from:
% http://www.LaTeXTemplates.com
%
% Original author:
% Trey Hunner (http://www.treyhunner.com/)
%
% Important note:
% This template requires the cv.cls file to be in the same directory as the
% .tex file. The cv.cls file provides the cv style used for structuring the
% document.
%
% Edited by Stefano Volpe (https://foxyseta.github.io):
% - prettify
% - remove colored text
% - add title, author
% - add content (my CV)
%
%%%%%%%%%%%%%%%%%%%%%%%%%%%%%%%%%%%%%%%%%%%%%%%%%%%%%%%%%%%%%%%%%%%%%%%%%%%%%%%%

%-------------------------------------------------------------------------------
%	PACKAGES AND OTHER DOCUMENT CONFIGURATIONS
%-------------------------------------------------------------------------------

\documentclass{cv} % Use the custom cv.cls style
\usepackage[dvipsnames]{xcolor}
\usepackage[hidelinks]{hyperref}
\usepackage{hyperref}
\hypersetup{
	pdftitle={Stefano Volpe's CV},
	pdfauthor={Stefano Volpe}
}
\usepackage{bookmark}

% Document margins
\usepackage[left=0.75in,top=0.6in,right=0.75in,bottom=0.1in]{geometry}
\newcommand{\tab}[1]{\hspace{.2667\textwidth}\rlap{#1}}
\newcommand{\itab}[1]{\hspace{0em}\rlap{#1}}
\name{Stefano Volpe} % Your name
\address{36, via Marconi \\ Bologna, 40122 (Italy)} % Your address
%\address{123 Pleasant Lane \\ City, State 12345} % Your secondary addess
\address{
	\href{https://github.com/foxyseta}{github.com/foxyseta} \\
	\href{https://foxyseta.github.io}{foxyseta.github.io}
}
\address{
	\href{mailto:stefano.volpe2@studio.unibo.it}{stefano.volpe2@studio.unibo.it}
} % Your phone number and email

\renewenvironment{rSection}[1]{
	\sectionskip
	% \textcolor{RoyalPurple}{\MakeUppercase{#1}}
	\par\refstepcounter{section}
	\sectionmark{#1}
	\addcontentsline{toc}{section}{\protect\numberline{\thesection}#1}
	\MakeUppercase{#1}
	\sectionlineskip
	\hrule
	\begin{list}{}{
			\setlength{\leftmargin}{1.5em}
		}
		\item[]
			}{
	\end{list}
}

\begin{document}

\begin{rSection}{Education}
	{\bf University of Bologna} \hfill
	\\ Bachelor in Computer Science \hfill {\em September 2020 - Present}
\end{rSection}

\begin{rSection}{Grants and Awards} \itemsep -2pt
	{Merit-based grant (\emph{University of Bologna})}\hfill {\em May 2022} \\
	{\emph{Romanae Disputationes} national contest, 1\textsuperscript{st} place}
	\hfill {\em February 2020} \\
	{Merit-based grant (\emph{Associazione Oriani})} \hfill {\em December 2019} \\
	{Merit-based grant (\emph{Associazione Oriani})} \hfill {\em December 2018}
\end{rSection}

\begin{rSection}{Lectures}
	{\bf University of Bologna} \hfill {\em Bologna, Italy}
	\\ \emph{An Introduction to Git} (peer-to-peer lecture)
	\hfill {\em November 2022}
\end{rSection}

\begin{rSection}{Industry Experience}
	{\bf devDept Software S.r.l.} \hfill {\em Borgo Panigale (BO), Italy}
	\\ Junior Programmer Analyst
	\hfill {\em October 2021 -- Present}

	{\bf Tozzi Group} \hfill {\em Mezzano (RA), Italy}
	\\ IT technician, full-stack developer (intern)
	\hfill {\em June 2019 -- July 2019}
\end{rSection}

\begin{rSection}{Service Experience}
	{\bf Italian Catholic Federation of University Students (Bologna group)}
	\hfill \\ Group President
	\hfill {\em February 2022 -- Present}

	{\bf Francis of Paola's parish} \hfill {\em Lugo (RA), Italy}
	\\ Sunday School Teacher
	\hfill {\em September 2017 -- May 2022}

	{\bf \emph{Liceo Scientifico ``A. Oriani''}} \hfill {\em Ravenna, Italy}
	\\ Volunteer Tutor
	\hfill {\em November 2018 -- February 2020}
\end{rSection}

\begin{rSection}{Interests} \itemsep -2pt
	Logic, theoretical computer science, mathematics, information theory, game
	theory, decision theory
\end{rSection}

\begin{rSection}{Languages} \itemsep -2pt
	\begin{itemize}
		\item English: I scored $8/9$ (TRF: $22IT010936VOLS010A$) at the IELTS
		      Academic test. IELTS itself considers this band to be ``borderline''
		      C1/C2.
		\item French: basic user. I studied French for three years in middle school.
		\item Italian: native speaker.
	\end{itemize}
\end{rSection}

\begin{rSection}{Skills}
	\begin{tabular}{ >{\bfseries}p{0.30\linewidth} p{0.65\linewidth} }
		Operating Systems & Linux systems                                     \\
		Code editors      & Vim, Neovim                                       \\
		Languages         & Agda, C, C++, C\#, Dockerfile, Haskell, Go, Java,
		JavaScript, \LaTeX, MatProg, PHP, Python, Scheme, SQL,
		TypeScript                                                            \\
	\end{tabular}
\end{rSection}

\end{document}
